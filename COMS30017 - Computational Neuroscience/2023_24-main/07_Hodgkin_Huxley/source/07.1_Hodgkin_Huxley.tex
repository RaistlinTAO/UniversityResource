\documentclass[12pt]{article}

\usepackage{graphicx}
\usepackage{pstricks}
\usepackage{amsmath}
\usepackage{listings}
\lstset{language=python}
\pagestyle{headings}
\markright{Computation Neuroscience - 07.1\_Hodgkin\_Huxley - Conor}



\usepackage{ifthen}
\newboolean{nopics}
\setboolean{nopics}{true}


\begin{document}

\subsection*{Gated channels}

\begin{figure}
  \ifthenelse{\boolean{nopics}}
             {\textsl{A very complex ball and stick diagram of a large molecule, it has a hole through the center and an obvious symmetry under a quarter turn.}}
             {
  \begin{center}
\includegraphics[width=8cm]{potassium_channel.png}
  \end{center}
  }
\caption{An open potassium channel, picture from  wikipedia, which in turn took it from the Protein Data Bank. \texttt{http://en.wikipedia.org/wiki/Potassium\_channel}}
\end{figure}

The nonlinear dynamics that neurons rely on to form spikes arise from
the \textbf{voltage-gated channels}; these are ion channels whose
conductance varies as the voltage varies. They are tiny molecular
machines which, crucially, are ion selective: only sodium ions can
pass through a sodium gate, only potassium ions through a potassium
gate. Each individual gate has a number of different gating states, we
will briefly examine this, but ultimately each one is either open or
closed, the overall smooth, though rapid, variation in these
conductances comes from average a large number of individual discrete
step-like changes as the individual gates open and close.

The potassium channel is a \textbf{persistent} gate; this is actually
a little complicated, but roughly speaking it has one type of closed
state and one type of open state; the sodium channel, which we will
look at after the potassium channel also has one open state, but it
has two types of closed states. 

\begin{figure}
  \ifthenelse{\boolean{nopics}}
  {\textsl{The horizonal axis runs from -100 mV to 20 mV, well it says V in the picture but that's wrong, the vertical from zero to one, there are three curves $n_\infty$ rises in a sigmoid fashion from zero to one, crossing 0.5 at around -55 mV, $m_\infty$ also rises from zero to one, but later crossing 0.5 at around -30 mV, $h_\infty$ in contrast falls from one to zero crossing 0.5 at around -65 mV }}
{\begin{center}
\include{asymp_vals}
\end{center}}
\caption{The asymptotic values of the gating probabilities.\label{fig:asymp_vals}}
\end{figure}

The potassium gate is actually composed of four independent subgates,
all these gates must be open to allow potassium ions through, but each
has its own independent dynamics. The membrane is usually modelled as
having overall potassium conductance
\begin{equation}
g_{K}=\bar{g}_Kn^4
\end{equation}
where $\bar{g}_K$ would be conductance if all the channels were open
and $n$ is the probability an individual subgate is open so $n^4$ is
the probability an individual gated channel is open. The dynamical equation for $n$ is quite complicated, it is of the standard form
\begin{equation}
\tau_n(V)\frac{dn(t)}{dt}=n_\infty(V)-n(t)
\end{equation}
If $\tau_n(V)$ and $n_\infty(V)$ where constant this would be simple,
$n(t)$ would decay to $n_\infty$ with a timescale of $\tau_n$, however
they aren't constants, they are functions of the membrane potential. 

A graph of $n_\infty(V)$ is shown in Fig.~\ref{fig:asymp_vals}. We can
see that $n$ is small when the voltage is near the resting value but
climbs towards one as $V$ increases. Now, $n$ isn't equal to
$n_\infty$, rather it decays towards it with a time constant given by
$\tau_n(V)$, but we can see that the potassium channels open as the
voltage increases. Before looking at how these play a role in spiking
we will look at the sodium gates. It is worth noting though the way
having four independent gates makes the dynamics crisper: if $n$ is
near zero, $n^4$ is very small indeed. This is illustrated in
Fig.~\ref{fig:nfour}.

We also need to discuss reversal potentials. You would expect the flow
of potassium to be determined by $g_KV$, but it isn't; because there
are more potassium ions inside the cell than outside they would flow
out even if $V=0$. In fact, as we discussed when looking at the
integrate and flow model, we assume this doesn't change Ohm's law, the
relationship between potential difference and current, rather, it just
changes the zero point:
\begin{equation}
I_K=g_K(E_K-V)
\end{equation}
where $E_K=-70$ mV, approximately, is called the reversal potential
and can be calculated using an equation called the Nernst equation.

\begin{figure}
  \ifthenelse{\boolean{nopics}}
             {\textsl{A very simple graph showing $n$ and $n^4$ for $n$ in zero to one, obviously $n^4$ stays near zero for longer.}}
             {
\begin{center}
\include{nfour}
\end{center}
}
\caption{The fourth power gives crisper behavior than $n$ itself would.\label{fig:nfour}}
\end{figure}


\begin{figure}
  \ifthenelse{\boolean{nopics}}
             {\textsl{The horizontal axis runs from -100 to 20 mV, the vertical from zero to 10 ms. There are three curves, $\tau_n$ starts near 5 ms, rises a small bit and then decays away to near zero, $\tau_h$ starts near 2.5 ms, rises sharply to around 8 ms near -60 mV and then decays to around 1 ms, $\tau_m$ is very small, never rising eyond 0.5 mS.}}
               {
\begin{center}
\include{tau_vals}
\end{center}
}
\caption{The time constants for the gating probabilities.\label{fig:tau_vals}}
\end{figure}


\begin{figure}
  \ifthenelse{\boolean{nopics}}
             {\textsl{There are drawings of the sodium and potassium channel, showing channels through a bilipid layer. In the sodium channel there are two states, one called closed where the channel is blocked by a bump on each side and one called open where these bumps are not there. For potassium there is also a sort of cork attached by a string near the channel, in addition to open and closed states resembling the states for sodium, with the closed state labelled `closed 1', there is an addition `closed 2' state where the bumps are not blocking the channel but the cork-like thing is.}}
               {
  \begin{center}
\includegraphics{ChannelsBlack.eps}
  \end{center}
  }
\caption{A cartoon of the two types of voltage gated channel we are
  discussing; the persistent potassium channels open as the voltage
  rise, the transient sodium channel opens and then closes
  again.\label{ChannelsBlack}}
\end{figure}



The sodium channel is called a transient channel because it has two
closed states and one open one; generally its dynamics during the
spike is
\begin{equation}
\mbox{closed I}\rightarrow \mbox{open}\rightarrow\mbox{closed II}
\end{equation}
After that there is a slower process of resetting. The part of the
gate that is closed to give the initial closed state is very like the
potassium gate, but with three subgates; the probability of these
subgates being open is usually called $m$; the other part, the gate
that closes to give the second closed state is different in that it is
not made of subgates, its probability of being open is usually called
$h$ and its asymptotic value, $h_\infty$, is near one for lower $V$
and near zero for larger. A cartoon of all this is given in Fig.~\ref{ChannelsBlack}. Fig.~\ref{fig:asymp_vals} includes graphs of $m_\infty(V)$
and $h_\infty(V)$. Finally, the reversal potential for sodium is
$E_{Na}=50$ mV. The sodium current is therefore
\begin{equation}
I_{Na}=g_{Na}(E_{Na}-V)
\end{equation}
with
\begin{equation}
g_{Na}=\bar{g}_{Na}m^3h
\end{equation}


All of this together gives the Hodgkin-Huxley equation, basically it
equates the rate of change of $V$ to a set of currents, the leak
current giving the roughly linear behavior below threshold we saw in
the integrate and fire model and the gated channels forming the
spike.
\begin{equation}
C_m\frac{dV}{dt}=\mbox{currents}
\end{equation}

We can now give a rough description of how spikes are formed. The time
constants $\tau$ for the three gating probabilities are given in
Fig.~\ref{fig:tau_vals}, these are quite complicated, but the key
thing is that $\tau_m$ is very small, no matter what the value of $V$
is. This means that $m$ stays very close to its asymptotic value
$m_\infty$. As $V$ approaches the threshold of about $-55$ mV, $m$
increases towards one, with $m^3$ increasing even more
dramatically. Opening the sodium gates allows sodium to flood the
cell, increasing the $V$ further and further opening the gates. This
gives the rapid upswing in voltage, the rising part of the spike. The
other two gating probabilities have slower dynamics and it takes $n$
and $h$ a while to catch up with $n_\infty$ and $h_\infty$. However,
as $h$ decreases, it closes the sodium gates again, preventing more
sodium getting in to the cell; $n$ increases opens the potassium
gates, potassium flows out reducing the $V$ again, back towards -70
mV. This gives the downswing of the spike. Afterwards everything
resets. An example spike is shown in Fig.~\ref{fig:HH_spike}.

\begin{figure}
  \ifthenelse{\boolean{nopics}}{\textsl{In A we see three spikes, the voltage rises sharply from around -55 mV to near 40 mV, in B there are curves for $n$, $h$ and $m$, the $m$ curve matches the spikes, rising and falling at the same time, the $n$ cuve is slower, only reaching its peak as the $m$ falls, $h$ falls dramatically during the spike.}}
    {
  \textbf{A}
\begin{center}
\include{HH_spike}
\end{center}
\textbf{B}
\begin{center}
\include{gates_spike}
\end{center}
}
\caption{Some spikes in the HH model. \textbf{A} shows the spikes produced by a standard HH model in response to a current input. \textbf{B} shows how the gating probabilities vary during the spike, in this graph $h$ and $m$ are labelled the wrong way around.\label{fig:HH_spike}}
\end{figure}

For a more accurate model further channels, and therefore further
currents, can be added, other sodium and potassium channels with
different dynamics, or a calcium channel. It is also common to
investigate models \lq{}between\rq{} the integrate and fire model and
the Hodgkin-Huxley equation which add some of the nonlinearity to the
integrate and fire dynamics.

\section*{Summary}

There are voltage depending channels in the cell membrane. These are modelled with the equation
\begin{equation}
  \tau_\ell\frac{d\ell}{dt}=\ell_\infty-\ell
\end{equation}
where $\ell$ is standing in for one of the so called gating probabilities. Here we consider the channels found in the giant axon of the squid. This has a potassium channel with conductance:
\begin{equation}
  g_{\text{K}}=\bar{g}_{\text{K}}n^4
\end{equation}
with gating probability $n$ and a sodium channel with conductance
\begin{equation}
  g_{\text{Na}}=\bar{g}_{\text{Na}}m^3h
\end{equation}
with gating probabilities $m$ and $h$. In the case of the the sodium
channel the timescale for $m$ is small so $m$ closely tracks
$m_\infty$; this rises towards one as voltage increase beyond the
threshold, so, provided $h$ is not close to zero, the sodium channel
opens as the voltage crosses zero, since the reversal potential for
sodium is high, this means sodium rushes in, causing $m$ to increase
still further. However, $h$ does not remain above zero, as the votage
increases $h_\infty$ falls to zero and $\tau_\infty$ decreases,
causing the sodium conductance to fall. At the same time, the
potassium conductance increases, since the reversal potential for
potassium is low this causes a current out of the cell, pushing the
voltage back down. Thus a spike is formed.


\begin{thebibliography}{99}
\bibitem{HodgkinHuxley1952}
Hodgkin, AL and Huxley HF (1952) 
\newblock"Propagation of electrical signals along giant nerve fibres." 
\newblock Proceedings of the Royal Society of London. Series B, Biological Sciences 140: 177-183.
\end{thebibliography}


\end{document}

