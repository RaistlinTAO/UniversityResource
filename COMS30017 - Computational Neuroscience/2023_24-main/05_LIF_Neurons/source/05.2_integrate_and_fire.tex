\documentclass[12pt]{article}
\usepackage{amsfonts, epsfig}

\usepackage{graphicx}
\usepackage{pstricks}
\usepackage{listings}

\usepackage{tikz}
%\usepackage{tikzscale}
%\usepackage{pgfplots}

%\pgfplotsset{compat=1.8}

\usepackage{fancyhdr}
\pagestyle{fancy}
\lfoot{\texttt{coms30127.github.io}}
\lhead{Computation Neuroscience - 05.1\_bucket\_equation - Conor}
\rhead{\thepage}
\cfoot{}


\usepackage{ifthen}
\newboolean{nopics}
\setboolean{nopics}{true}


\begin{document}



\subsection*{Electrical properties of a neuron}
The potential inside a neuron is lower than the potential on the
outside; this difference is created by ion pumps, small molecular
machines that use energy to pump ions across the membrane seperating
the inside and outside of the cell. One typical ion pump is
Na+/K+-ATPase (Sodium-potassium adenosine triphosphatase); this uses
energy in the form of ATP, the energy carrying molecule in the body,
and through each cycle it moves three sodium ions out of the cell and
two potassium ions into the cell. Since both sodium and potassium ions
have a charge of plus one, this leads to a net loss of one atomic
charge to the inside of the cell lowering its potential. It also
creates an excess of sodium outside the cell and an excess of
potassium inside it. We will return to these chemical imbalances
later. The potential difference across the membrane is called the
\textbf{membrane potential}. At rest a typical value of the membrane
potential is $E_L=-70 $mV. It is useful to remember that the excessive
sodium is outside the cell and potassium inside; I think of islands
which are surrounded by salty water, as in Fig.~\ref{in_out}.


\begin{figure}
    \ifthenelse{\boolean{nopics}}
    {\textsl{A map of the UK is marked with u\textbf{K}$^+$ in the land area and \textbf{Na}$^+$Cl$*-$ in the sea area, for comic effect the `UK' as actually just England and Wales.} 
}
    {
        \begin{center}
      \includegraphics[width=6cm]{in_out_future}
        \end{center}
        }
\caption{A cartoon to help you remember that there is more sodium outside the cell and more potassium inside.\label{in_out}}
\end{figure}


\subsection*{Spikes}

So the summary version of what happens in neuons is that
\textbf{synapses} cause a small increase or decrease in the voltage;
\textbf{excitatory synapses} cause an increase, \textbf{inhibitory
  synapses} a decrease. This drives the internal voltage dynamics of
the cell, these dynamics are what we will learn about here. If the
voltage exceeds a threshold, say $V_T=-55$ mV there is a nonlinear
cascade which produces a \textbf{spike} or \textbf{action potential},
a spike in voltage 1-2 ms wide which rises above 0 mV before, in the
usual description, falling to a reset value of $V_R=-65$ mV, the cell
then remains unable to produce another spike for a \textbf{refractory
  period} which may last about 5 ms. We will examine how spikes are
formed later, this involves the nonlinear dynamics of ion channels in
the membrane; first though we will consider the integrate and fire
model which ignores the details of how spikes are produced and
simplifies the voltage dynamics.

\subsection*{The bucket-like equation for neurons}

We will now try to extend the bucket-like equation we looked at before
so that it applies to neurons. First off we replace $h$, the height of
the water, by $V$ the voltage in the cell and $C$ will be replaced by
$C_m$, the capacitance of the membrane, the amount of electrical
charge that can be stored at the membrane is $C_mV$. The amount of
electrical charge is the analogue of the volume of water. Thus,
voltage is like height, charge is like the amount of water.

The leak is a bit more complicated, because of the chemical gradients,
that is the effects of the differing levels of ions inside and outside
the cell along and their propensity to diffuse, the voltage at which
there is no leaking of charge is not zero, it is $E_L=-70 $mV,
roughly. This is an important aspect of how neurons behave, and one we
will encounter again looking at the Hodgkin-Huxley equation: you might
at first expect that if the voltage inside the cell was, say, -60 mV
then even if there was a high conductivity for potassium at the
membrane, the potassium ions would stay in the cell: they are positive
ions after all and so a negative voltage means the electrical force is
attracting them to the inside of the cell. However, this isn't quite
what happens, there is a high concentration of potassium inside the
cell and because of the random motion of particles associated with
temperature, these have a tendency to diffuse, that is to increase the
entropy of the situation by spreading out. It takes a force to
counteract this. This is the reversal potential, $E_L$, the voltage
required for zero current even if there is some conductivity. It turns
out that the normal Ohm's law applies around the reversal potential so
that the current out of the cell is proportional to $V-E_L$.

$G$ is now $G_m$, a conductance, measuring the porousness of the
membrane to the flow of ions, in other words, it gives the constant of proportionality for the leak current: the leak current out of the cell is
$G_m(V-E_L)$. We actually divide across by the conductance, and write
$R_m=1/G_m$, the resistance. Finally, we write $\tau_m=C_m/G_m$ to get
\begin{equation}
\tau_m\frac{dV}{dt}=E_L-V+R_mI
\end{equation}
$I$ might end up being synaptic input, but traditionally we write the
equation to match the \textsl{in vivo} experiment where $I$ is an
injected current from an electrode, so we write $I_e$, \lq{}e\rq{} for
electrode. $\tau_m$ is a time constant, using the notation of
dimensional analysis we have $[\tau_m]=T$. To check this note that the
units of capacitance are charge per voltage: $[C_m]=QV^{-1}$, the
units of resistance is voltage per current $[R_m]=VI^{-1}$ and current
is charge per time, $[I]=QT^{-1}$ so $[C_mR_m]=T$, time.

The equation above leaves out the possibility that there are other
non-linear changes in the currents through the membrane as $V$
changes. This is a problem since there are other non-linear changes in
the currents through the membrane as $V$ changes. The equation above
leaves these out, in fact, the nonlinear effects are strongest for
values of $V$ near where a spike is produced, so one approach is to
use the linear equation unless $V$ reaches a threshold value and then
add a spike \lq{}by hand\rq{}. This has the effect of changing the
voltage to a reset value, this mimics what happens in the neuron, or
in the Hodgkin Huxley model which we will look at next and which
includes the full non-linear dynamics which makes the spike. Anyway,
in summary
\begin{itemize}
\item $V$ satisfies
\begin{equation}
\tau_m\frac{dV}{dt}=E_L-V+R_mI_e
\end{equation}
\item If $V\ge V_T$ a spike is recorded and the voltage is set to a
  reset value $V_R$.
\end{itemize}
The reset value, the voltage after the spike is often set equal to the
 leak potential. This is the \textbf{leaky integrate and fire model}, a surprisingly
old model first introduced in \cite{Lapicque1907a}. It lacks lots of
the details important in the dynamics of neurons, but is useful and
often used for modeling the behavior of large neuronal networks or for
exploring ideas about neuronal computation in a relatively
straight-forward setting.

This model is easy to solve; if $I_e$ is constant we have already
solved it above, up to messing around with constants:
\begin{equation}
V(t)=E_L+R_mI_e+[V(0)-E_L-R_mI_e]e^{-t/\tau_m}
\end{equation}
If $I_e$ is not
constant it may still be possible to solve the equation, but in any
case the equation can be solved numerically on a computer. An example
in given in Fig.~\ref{v_i_f}.

\begin{figure}
  \begin{center}
    \ifthenelse{\boolean{nopics}}
    {\textsl{In this graph the horizontal axis is labelled $t$ (ms), the vertical $V$ $mV$. The former runs from zero to 100, the latter from -70 to 20. There are two curves, one marked $R_mI_e=12$ mV rises from -70 and is soon indistinguishable from a constant lines at -58 mV; the one marked $R_mI_e=16$ mV rises to $-55$ mV, then there is a spike to 20 mV and the curve starts again at $-70$ mV. This happens three times}}
{  \include{v_i_f}
 } 
\end{center}
\caption{An integrate and fire neuron with different inputs. For
  $R_mI=12 $mV the voltage relaxes towards the equilibrium value
  $V=E_L+R_mI_e=-58$ mV. It never reaches the threshold value of
  $V_T=-55 $mV. For $R_mI=16$ mV the voltage reaches threshold and so
  there is a spike; the spike is added by hand, in this case by
  setting $V$ to $20$ mV for one time step. The voltage is then
  reset. Here $\tau_m=10$ ms.\label{v_i_f}}
\end{figure}

One thing to notice is that there are no spikes for low values of the current. Looking at the equation 
\begin{equation}
\tau_m\frac{dV}{dt}=E_L-V+R_mI_e
\end{equation}
so the equilibrium value for constant $I_e$, the value where $V$ stops changing, is
\begin{equation}
\bar{V}=E_L+R_mI_e
\end{equation}
Now if this value $\bar{V}>V_T$ then as the neuron voltage increased
towards its equilibrium value, $\bar{V}$, it would reach the
threshold, $V_T$, and spike. Hence, if $\bar{V}>V_T$ the neuron will
spike repeatedly.  However if $\bar{V}<V_T$ then the neuron will not
spike for that input because it will never reach threshold. 

In fact, we can work out the curve that represents the firing rate as
a function of the current; this is the called the f-I curve and is shown in Fig.~\ref{f_i_curve}. In the model
\begin{equation}
\tau_m\frac{dV}{dt}=E_L-V+R_mI_e
\end{equation}
which we can solve from our study of odes, it gives
\begin{equation}
V(t)=E_L+R_mI_e+[V(0)-E_L-R_mI_e]e^{-t/\tau_m}
\end{equation}
so if the neuron has spiked and is reset at time $t=0$ and reaches
threshold at time $t=T$, assume $V_R=E_L$ we have
\begin{equation}
V_T=E_L+R_mI_e-R_mI_ee^{-T/\tau_m}
\end{equation}
so 
\begin{equation}
e^{-T/\tau_m}=\frac{E_L+R_mI_e-V_T}{R_mI_e}
\end{equation}
Taking the log of both sides we get
\begin{equation}
T=\tau_m\log\left[\frac{R_mI_e}{E_L+R_mI_e-V_T}\right]
\end{equation}
Finally, this is the time between spikes, so the frequency is one over this. It is only defined for $R_mI_e>V_T-E_L$, below that there is no spiking and the frequency is zero. The actual gnuplot command used to make the figure was\\
\texttt{plot [0:22] x>15 ? 1/(.01*log(x/(x-15))) : 0}


\begin{figure}
  \begin{center}
    \ifthenelse{\boolean{nopics}}
    {\textsl{In this graph the vertical axis is marked `firing rate in Hz' and goes from zero to 80, the horizontal $R_mI_e$ and goes from zero to 20. The curve is zero until $R_mI_e=15$, after that it rises, sharply at first, almost going straight up, and then steeply but not quite as steep to reach 80 near $R_mI_e=20$.} }
{\include{f_i_curve}
}
  \end{center}
\caption{The firing rate, that is spikes per second, for the integrate
  and fire neuron with different constant inputs with $\tau_m=10$ ms,
  $V_T=-55$ mV and both the leak and reset given by $-70$ mV. Notice
  how there is no firing until a threshold is reached and after that
  the firing increases very quickly. \label{f_i_curve}}
\end{figure}


We have been ignoring the refractory period, this can easily be added
to the model. The easiest approach is just to adjust the rule for
spiking by specifying the voltage remains fixed at $V_r$ for some
time, $T_r$, corresponding to the refractory period. In fact, the
refractory period tends to be divided in the \textsl{absolute
  refractory period} when the neuron can't spike at all and a
\textsl{relative refractory period} when spiking is just
harder. Holding $V$ at $V_r$ models an absolute refractory period, to
model a relative refractory period a negative current can be added to
the equation.

In fact, neurons tend not to have a regular firing rate; they often,
for example, show spike rate adaptation. This means that the firing
rate decreases even if the incoming current remains constant. As we
have seen this isn't modelling in the integrate and fire model, in the
integrate and fire model the response to a constant current is a
constant firing rate. The model can be extended to include it. For
example a slow potassium current could be added, a slow potassium
current is a current that would increase every time there is a spike
and decrease between spikes. Since a potassium current reduces the
voltage inside the cell, it decreases the firing rate; thus, if the
cell fires quickly the potassium current increases, decreasing the
firing rate until the increase in the potassium when there is a spike
balances the decrease between spikes. These slow currents are common,
they may help protect the cell from `exhaustion' by reducing its spike
rate for ongoing stimuli and they may play a computation role, helping
the brain infer the nature of stimuli across timescales. The `face of
Jesus' illusion demonstrates the existence of these currents.

A simple approach, which would approximate adding a potassium would be
to just add a second input $u$
\begin{equation}
  \tau\frac{dV}{dt}=R_mI_e-V+u
\end{equation}
where
\begin{equation}
  \tau_u\frac{du}{dt}=-u
\end{equation}
where there is an additional rule that $u\leftarrow u+\delta u$
whenever there is a spike. Thus if $u$ is negative it will reduce the
equilibrium value of $V$, reducing the spike rate if the cell is
spiking. $u$ will decay towards zero with a timescale $\tau_u$ but
spiking will change it by $\delta u$. In the case we have discussed
here, where we want to model a reduction in spiking $\delta u$ would
be negative and $\tau_u$ would have a moderate timescale, perhaps 200
ms.

\section*{Summary}

Chemical gradients mean that current flows across the membranes of the
cell are not zero for zero potential There is a reversal potential for
each type of ion, this is the value of the potential for which no
current flows. For our purposes here we consider the leak potential;
at lower voltages most of the channels that allow ions to pass through
the membrane are closed, this means the conductance is near zero. The
are some open channels, mostly for potassium and the corresponding
reversal potential is called the leak potential $E_l$ while the
conductance is $g_l$. $E_l$ is often take to be somewhere around -80
mV. Using Ohms law the equation for the voltage in a cell is then
\begin{equation}
  C_m\frac{dV}{dt}=g_l(E_l-V)+I_e
\end{equation}
where $C_m$ is the capacitance of the membrane and $I_e$ is the input current. This is also written
\begin{equation}
  \tau_m\frac{dV}{dt}=E_l-V+R_mI_e
\end{equation}
where $\tau_m$ is a timescale, often around 12-20 ms and
$R_m=1/g_l$. However, this linear equation does not include the
nonlinear dynamics that supports spiking, we add spiking `by hand'
with a rule that says if $V>V_t$, a threshold value, usually about -55
mV, then there is a spike and the voltage is reset to a reset value
$V_r$ often set equal to, or a little greater than, $E_l$. This is the
leaky integrate and fire equation. We know something of what the
solutions look like from the leaky bucket equation considered before,
but now there are spikes. We can see that if the equilibrium value
$E_l+R_mI_e$ is greater than $V_t$ the neuron will spike regularly, if
it is less, it won't spike at all. The equation leaves out the actual
spiking dynamics, it also ignores the actual spatial organization of
the neuron, treating it as a point. It does not include some other
aspects of neuronal spiking like the refractory period and slow
currents, though these can be added, at least approximately.


\begin{thebibliography}{99}
\bibitem{Lapicque1907a}
Lapicque, L. (1907). 
\newblock Recherches quantitatives sur l'excitation électrique des nerfs traitée comme une polarisation. 
\newblock J. Physiol. Pathol. Gen, 9:620--635.
\end{thebibliography}

\end{document}
